\documentclass[en, device=normal]{elegantnote}

\usepackage[]{amssymb}

\title{Introduction to high energy physics}

\author{Grant}

\date{\today}

\begin{document}
\maketitle

\section{Quarks and leptons}

\thispagestyle{empty}

\subsection{Preamble}

Nothing to say.

\subsubsection{Why high energies?}

Particle physics deals with the study of the elementary constituents of matter.
We need to talk about "pointlike". It is a conception depends on the spatial resolution 
of the probe used to investigate possible structure. The resolution is $\Delta r$ if 
two points in an object can just be resolved as separate when they are a distance $\Delta r$ 
apart.

Assuming the probing beam itself consists of pointlike particles, the resolution is limited by 
the de Broglie wavelength of these particles, which is $\lambda = h/p$ where $p$ is the 
beam momentum and $h$ is Planck's constant. Thus beams of high momentum have short 
wavelengths and can have high resolution.

In an optical micro scope, the resolution is given by

$$\Delta r\simeq\lambda/ \sin\theta$$

where $\theta$ is the angular aperture of the light beam used to view the structure of 
an object.

Substituting the de Broglie relation, the resolution becomes 

$$\Delta\simeq\frac{\lambda}{\sin\theta}=\frac{h}{p\sin\theta}$$

so that $\Delta r$ is inversely proportional to the momentum transferred to the photons, 
or other particles in an incident beam, when these are scattered by the target. Therefore, 
the first reason is resolution.

The second reason is that many of the elementary particles are extremely massive and the 
energy required to create them is correspondingly large.

\subsubsection{Units in high energy physics}

\begin{table}[h]
  \centering
  \caption{Units in high energy physics(1)}
    \begin{tabular}{lll}
    \toprule
    Quantity & High energy unit & Value in SI units \\
    \midrule
    length & $1fm$ & $10^{-15}m$ \\
    energy & $1GeV=10^9eV$ & $1.602\times 19^{-10}J$ \\
    mass, $E/c^2$ & $1GeV/c^2$ & $1.78\times 10^{-27}kg$ \\
    $\hbar=h/(2\pi)$ & $6.588\times 10^{-25}GeV\cdot s$ & $1.055\times 10^{-34}J\cdot s$ \\
    $c$ & $2.998\times 10^{23}fm\cdot s^{-1}$ & $2.998\times 10^8m\cdot s^{-1}$ \\
    $\hbar c$ & $0.1975GeV\cdot fm$ & $3.162\times 10^{-26}J\cdot m$ \\
    \bottomrule
    \end{tabular}
\end{table}

\begin{table}[h]
  \centering
  \caption{Units in high energy physics(2)}
  \begin{tabular}{ll}
    \hline
    \multicolumn{2}{l}{natural units, $\hbar=c=1$} \\
    mass, $Mc^2/c^2$ & $1GeV$ \\
    length, $\hbar c/(Mc^2)$ & $1GeV^{-1}=0.1975fm$ \\
    time, $\hbar c/(Mc^3)$ & $1GeV^{-1}=6.59\times 10^{-25}s$ \\
    \hline
    \multicolumn{2}{l}{Heaviside-Lorentz units, $\epsilon_0=\mu_0=\hbar=c=1$} \\
    fine structure constant & $\alpha=e^2/(4\pi)=1/137.06$ \\
    \hline
  \end{tabular}
\end{table}

In the SI system the unit electric charge, $e$, is measured in coulombs and the fine structure 
constant is given by

$$\alpha=\frac{e^2}{4\pi\epsilon_0\hbar c}\simeq\frac{1}{137}$$

Here $\epsilon_0$ is the permittivity of free space, while its permeability is defined as $\mu_0$, 
such that $\epsilon_0\mu_0=1/c^2$.

For interaction in general, such units are not useful and we can define $e$ in Heaviside-Lorentz 
units, which require $\epsilon_0=\mu_0=\hbar=c=1$, so that

$$\alpha=\frac{e^2}{4\pi}\simeq\frac{1}{137}$$

with similar definitions that relate charges and coupling constants analogous to $\alpha$ in 
the other interactions.

\subsubsection{Relativistic transformations}

The relativistic relation between total energy $E$, the vector 3-momentum $\textbf{p}$(with Cartesian components $p_x,p_y,p_z$) 
and the rest mass m for a free particle is 

$$E^2=\textbf{p}^2c^2+m^2c^4$$

or, in units with $c=1$

$$E^2=\textbf{p}^2+m^2$$

The components $p_x,p_y,p_z,E$ can be written as components of an energy-momentum 4-vector 
$p_\mu$, where $\mu=1,2,3,4$. In the Minkowski convention, the three momentum(or space) components 
are taken to be real and the energy (or time) component to be imaginary, as follows: 

$$p_1=p_x,\quad p_2=p_y,\quad p_3=p_z,\quad p_4=iE$$

so that 

$$p^2=\sum_\mu p^2_\mu=\textbf{p}^2-E^2=-m^2$$

Thus $p^2$ is a relativistic invariant. Its value is $-m^2$, where $m$ is the rest mass, 
and clearly has the same value in all reference frames. If $E,\textbf{p}$ refer to the values measured in the lab 
frame $\Sigma$ then those in another frame, say $\Sigma'$, moving along the $x$-axis with 
velocity $\beta c$ are found from the Lorentz transformation, given in matrix form by 

$$p_\mu'=\sum_{\nu=1}^4\alpha_{\mu\nu}p_\nu$$

where

$$\alpha_{\mu\nu}=
\left\vert
\begin{array}{cccc}
\gamma & 0 & 0 & i\beta\gamma \\
0 & 1 & 0 & 0 \\
0 & 0 & 1 & 0 \\
-i\beta\gamma & 0 & 0 & \gamma \\
\end{array}
\right\vert$$

and $\gamma=1/\sqrt{1-\beta^2}$. Thus 

$$
\begin{aligned}
  p_1'&=\gamma p_1+i\beta\gamma p_4 \\
  p_2'&=p_2 \\
  p_3'&=p_3 \\
  p_4'&=-i\beta\gamma p_1+\gamma p_4 \\
\end{aligned}
$$

In terms of energy and momentum 

$$\begin{aligned}
  p'_x&=\gamma(p_x-\beta E) \\
  p'_y&=p_y \\
  p'_z&=p_z \\
  E'&=\gamma(E-\beta p_x) \\
\end{aligned}$$

with $\textbf{p}'^2-E'^2=-m^2$.

The above transformations apply equally to the space-time coordinates, making the 
replacements $p_1\rightarrow x_1(=x),p_2\rightarrow x_2(=y),p_3\rightarrow x_3(=z),p_4\rightarrow x_4(=it)$.

The 4-momentum squared above is an example of a Lorentz scalar, i.e. the invariant 
scalar product of the two 4-vectors. Another example is the phase of a plane wave, 
which determines whether it is at a crest or a trough and which must be the same for 
all observers. With \textbf{k} and $\omega$ as the propagation vector and the angular 
frequency, and in units $\hbar=c=1$

$$phase=\textbf{k}\cdot\textbf{x}-\omega t=\textbf{p}\cdot\textbf{x}-Et=\sum p_\mu x_\mu$$

The Minkowski notation used here for 4-vectors defines the metric, namely the square of 
the 4-vector momentum $p=(\textbf{p},iE)$ so that 

$$metric=(4-momentum)^2=(3-momentum)^2-(energy)^2$$

In analogy with the space-time components, the components $p_{x,y,z}$ of 3-momentum 
are said to be spacelike and the energy component $E$, timelike. Thus, if $q$ denotes 
the 4-momentum transfer in a reaction, i.e. is $q=p-p'$ where $p,p'$ are the initial 
and final 4-momenta, then $q^2>0$ means spacelike, otherwise, timelike.

A different notation is used in texts on field theory. These avoid the use of the imaginary 
fourth component and introduce the negative sign via the metric tensor $g_{\mu\nu}$. 
The scalar product of 4-vectors A and B is then defined as 

$$AB=g_{\mu\nu}A_\mu B_\nu=A_0B_0-\textbf{A}\cdot\textbf{B}$$

where all the components are real. Here $\mu,\nu =0$ stand for the energy(time) component 
and $\mu,\nu=1,2,3$ for the momentum(space) components, and 

$$g_{00}=+1,\quad g_{11}=g_{22}=g_{33}=-1,\quad g_{\mu\nu}=0\ for\ \mu\neq\nu$$

This metric results in Lorentz scalars with sign opposite to those using the Minkowski 
convention, so that a spacelike(timelike) 4-momentum has $q^2<0$($q^2>0$) respectively.

Sometimes, to avoid writing negative quantities, re-definitions have to be made. In deep 
inelastic electron scattering, $q^2$ is spacelike and negative, and in discussing such 
processes it has become common to define the positive quantity $Q^2=-q^2$.

\subsubsection{Fixed-target and colliding beam accelerators}

Consider the energy available for particle creation in fixed-target and in colliding-beam 
accelerators.

Suppose an incident particle of mass $m_A$, total energy $E_A$ and momentum $\textbf{p}_A$ 
hits a target particle of mass $m_B$, energy $E_B$, momentum $\textbf{p}_B$. The total 
4-momentum, squared, of the system is 

$$p^2=(\textbf{p}_A+\textbf{p}_B)^2-(E_A+E_B)^2=-m_A^2-m_B^2+2\textbf{p}_A\cdot\textbf{p}_B-2E_AE_B$$

The centre-of-momentum system(cms) is defined as the reference frame in which the total 
3-momentum is zero. If the total energy in the cms is denoted $E^*$, then we also have $p^2=-E^{*2}$.

Suppose first of all that the target particle is at rest in the laboratory(lab) system, so that $\textbf{p}_B=0$ 
and $E_B=m_B$. Then 

$$E^{*2}=-p^2=m_A^2+m_B^2+2m_BE_A$$

Secondly, suppose that the incident and target particles travel in opposite directions. 
Then, with $p_A$ and $p_B$ denoting the absolute values of the 3-momenta, the above 
equation gives 

$$E^{*2}=2(E_AE_B+p_Ap_B)+(m_A^2+m_B^2)\simeq 4E_AE_B$$

if $$m_A,m_B\ll E_A,E_B$$. This result is for a head-on collision. For two beams crossing at an 
angle $\theta$, the result would be $E^{*2}=2E_AE_B(1+\cos\theta)$.

Note that the cms energy available for new particle creation in a collider with equal energies 
$E$ in the two beams rises linearly with $E$, i.e. $E^*\simeq 2E$, while for a fixed-target 
machine the cms energy rises as the square root of the incident energy, $E^*\simeq\sqrt{2m_BE_A}$. 
Therefore, the highest possible energies for creating new particles are to be found at 
colliding-beam accelerators.

\subsection{The Standard Model of particle physics}

\subsubsection{The fundamental fermions}

\subsubsection{The interactions}

\subsubsection{Limitations of the Standard Model}

\subsection{Particle classification: fermions and bosons}

\subsection{Particles and antiparticles}

\subsection{Free particle wave equations}

\subsection{Helicity states: helicity conservation}

\subsection{Lepton flavours}

\subsection{Quark flavours}

\subsection{The cosmic connection}

\subsubsection{Early work in cosmic rays}

\subsubsection{Particle physics in cosmology}

\section{Interactions and fields}

\section{Invariance principles and conservation laws}

\section{Quarks in hadrons}

\section{Lepton and quark scattering}

\section{Quark interactions and QCD}

\section{Weak interactions}

\section{Electroweak interactions and the Standard Model}

\section{Physics beyond the Standard Model}

\section{Particle physics and cosmology}

\section{Experimental methods}

\end{document}
