\documentclass[en, device=normal]{elegantnote}

\usepackage[]{amssymb}

\title{Introduction to high energy physics}

\author{Grant}

\date{\today}

\begin{document}
\maketitle

\section{Quarks and leptons}

\thispagestyle{empty}

\subsection{Preamble}

Nothing to say.

\subsubsection{Why high energies?}

Particle physics deals with the study of the elementary constituents of matter.
We need to talk about "pointlike". It is a conception depends on the spatial resolution 
of the probe used to investigate possible structure. The resolution is $\Delta r$ if 
two points in an object can just be resolved as separate when they are a distance $\Delta r$ 
apart.

Assuming the probing beam itself consists of pointlike particles, the resolution is limited by 
the de Broglie wavelength of these particles, which is $\lambda = h/p$ where $p$ is the 
beam momentum and $h$ is Planck's constant. Thus beams of high momentum have short 
wavelengths and can have high resolution.

In an optical micro scope, the resolution is given by

$$\Delta r\simeq\lambda/ \sin\theta$$

where $\theta$ is the angular aperture of the light beam used to view the structure of 
an object.

Substituting the de Broglie relation, the resolution becomes 

$$\Delta\simeq\frac{\lambda}{\sin\theta}=\frac{h}{p\sin\theta}$$

so that $\Delta r$ is inversely proportional to the momentum transferred to the photons, 
or other particles in an incident beam, when these are scattered by the target. Therefore, 
the first reason is resolution.

The second reason is that many of the elementary particles are extremely massive and the 
energy required to create them is correspondingly large.

\subsubsection{Units in high energy physics}

\begin{table}[h]
  \centering
  \caption{Units in high energy physics(1)}
    \begin{tabular}{lll}
    \toprule
    Quantity & High energy unit & Value in SI units \\
    \midrule
    length & $1fm$ & $10^{-15}m$ \\
    energy & $1GeV=10^9eV$ & $1.602\times 19^{-10}J$ \\
    mass, $E/c^2$ & $1GeV/c^2$ & $1.78\times 10^{-27}kg$ \\
    $\hbar=h/(2\pi)$ & $6.588\times 10^{-25}GeV\cdot s$ & $1.055\times 10^{-34}J\cdot s$ \\
    $c$ & $2.998\times 10^{23}fm\cdot s^{-1}$ & $2.998\times 10^8m\cdot s^{-1}$ \\
    $\hbar c$ & $0.1975GeV\cdot fm$ & $3.162\times 10^{-26}J\cdot m$ \\
    \bottomrule
    \end{tabular}
\end{table}

\begin{table}[h]
  \centering
  \caption{Units in high energy physics(2)}
  \begin{tabular}{ll}
    \hline
    \multicolumn{2}{l}{natural units, $\hbar=c=1$} \\
    mass, $Mc^2/c^2$ & $1GeV$ \\
    length, $\hbar c/(Mc^2)$ & $1GeV^{-1}=0.1975fm$ \\
    time, $\hbar c/(Mc^3)$ & $1GeV^{-1}=6.59\times 10^{-25}s$ \\
    \hline
    \multicolumn{2}{l}{Heaviside-Lorentz units, $\epsilon_0=\mu_0=\hbar=c=1$} \\
    fine structure constant & $\alpha=e^2/(4\pi)=1/137.06$ \\
    \hline
  \end{tabular}
\end{table}

In the SI system the unit electric charge, $e$, is measured in coulombs and the fine structure 
constant is given by

$$\alpha=\frac{e^2}{4\pi\epsilon_0\hbar c}\simeq\frac{1}{137}$$

Here $\epsilon_0$ is the permittivity of free space, while its permeability is defined as $\mu_0$, 
such that $\epsilon_0\mu_0=1/c^2$.

For interaction in general, such units are not useful and we can define $e$ in Heaviside-Lorentz 
units, which require $\epsilon_0=\mu_0=\hbar=c=1$, so that

$$\alpha=\frac{e^2}{4\pi}\simeq\frac{1}{137}$$

with similar definitions that relate charges and coupling constants analogous to $\alpha$ in 
the other interactions.

\subsubsection{Relativistic transformations}

The relativistic relation between total energy $E$, the vector 3-momentum $\textbf{p}$(with Cartesian components $p_x,p_y,p_z$) 
and the rest mass m for a free particle is 

$$E^2=\textbf{p}^2c^2+m^2c^4$$

or, in units with $c=1$

$$E^2=\textbf{p}^2+m^2$$

The components $p_x,p_y,p_z,E$ can be written as components of an energy-momentum 4-vector 
$p_\mu$, where $\mu=1,2,3,4$. In the Minkowski convention, the three momentum(or space) components 
are taken to be real and the energy (or time) component to be imaginary, as follows: 

$$p_1=p_x,\quad p_2=p_y,\quad p_3=p_z,\quad p_4=iE$$

so that 

$$p^2=\sum_\mu p^2_\mu=\textbf{p}^2-E^2=-m^2$$

Thus $p^2$ is a relativistic invariant. Its value is $-m^2$, where $m$ is the rest mass, 
and clearly has the same value in all reference frames. If $E,\textbf{p}$ refer to the values measured in the lab 
frame $\Sigma$ then those in another frame, say $\Sigma'$, moving along the $x$-axis with 
velocity $\beta c$ are found from the Lorentz transformation, given in matrix form by 

$$p_\mu'=\sum_{\nu=1}^4\alpha_{\mu\nu}p_\nu$$

where

$$\alpha_{\mu\nu}=
\left\vert
\begin{array}{cccc}
\gamma & 0 & 0 & i\beta\gamma \\
0 & 1 & 0 & 0 \\
0 & 0 & 1 & 0 \\
-i\beta\gamma & 0 & 0 & \gamma \\
\end{array}
\right\vert$$

and $\gamma=1/\sqrt{1-\beta^2}$. Thus 

$$
\begin{aligned}
  p_1'&=\gamma p_1+i\beta\gamma p_4 \\
  p_2'&=p_2 \\
  p_3'&=p_3 \\
  p_4'&=-i\beta\gamma p_1+\gamma p_4 \\
\end{aligned}
$$

In terms of energy and momentum 

$$\begin{aligned}
  p'_x&=\gamma(p_x-\beta E) \\
  p'_y&=p_y \\
  p'_z&=p_z \\
  E'&=\gamma(E-\beta p_x) \\
\end{aligned}$$

with $\textbf{p}'^2-E'^2=-m^2$.

The above transformations apply equally to the space-time coordinates, making the 
replacements $p_1\rightarrow x_1(=x),p_2\rightarrow x_2(=y),p_3\rightarrow x_3(=z),p_4\rightarrow x_4(=it)$.

The 4-momentum squared above is an example of a Lorentz scalar, i.e. the invariant 
scalar product of the two 4-vectors. Another example is the phase of a plane wave, 
which determines whether it is at a crest or a trough and which must be the same for 
all observers. With \textbf{k} and $\omega$ as the propagation vector and the angular 
frequency, and in units $\hbar=c=1$

$$phase=\textbf{k}\cdot\textbf{x}-\omega t=\textbf{p}\cdot\textbf{x}-Et=\sum p_\mu x_\mu$$

The Minkowski notation used here for 4-vectors defines the metric, namely the square of 
the 4-vector momentum $p=(\textbf{p},iE)$ so that 

$$metric=(4-momentum)^2=(3-momentum)^2-(energy)^2$$

In analogy with the space-time components, the components $p_{x,y,z}$ of 3-momentum 
are said to be spacelike and the energy component $E$, timelike. Thus, if $q$ denotes 
the 4-momentum transfer in a reaction, i.e. is $q=p-p'$ where $p,p'$ are the initial 
and final 4-momenta, then $q^2>0$ means spacelike, otherwise, timelike.

A different notation is used in texts on field theory. These avoid the use of the imaginary 
fourth component and introduce the negative sign via the metric tensor $g_{\mu\nu}$. 
The scalar product of 4-vectors A and B is then defined as 

$$AB=g_{\mu\nu}A_\mu B_\nu=A_0B_0-\textbf{A}\cdot\textbf{B}$$

where all the components are real. Here $\mu,\nu =0$ stand for the energy(time) component 
and $\mu,\nu=1,2,3$ for the momentum(space) components, and 

$$g_{00}=+1,\quad g_{11}=g_{22}=g_{33}=-1,\quad g_{\mu\nu}=0\ for\ \mu\neq\nu$$

This metric results in Lorentz scalars with sign opposite to those using the Minkowski 
convention, so that a spacelike(timelike) 4-momentum has $q^2<0$($q^2>0$) respectively.

Sometimes, to avoid writing negative quantities, re-definitions have to be made. In deep 
inelastic electron scattering, $q^2$ is spacelike and negative, and in discussing such 
processes it has become common to define the positive quantity $Q^2=-q^2$.

\subsubsection{Fixed-target and colliding beam accelerators}

Consider the energy available for particle creation in fixed-target and in colliding-beam 
accelerators.

Suppose an incident particle of mass $m_A$, total energy $E_A$ and momentum $\textbf{p}_A$ 
hits a target particle of mass $m_B$, energy $E_B$, momentum $\textbf{p}_B$. The total 
4-momentum, squared, of the system is 

$$p^2=(\textbf{p}_A+\textbf{p}_B)^2-(E_A+E_B)^2=-m_A^2-m_B^2+2\textbf{p}_A\cdot\textbf{p}_B-2E_AE_B$$

The centre-of-momentum system(cms) is defined as the reference frame in which the total 
3-momentum is zero. If the total energy in the cms is denoted $E^*$, then we also have $p^2=-E^{*2}$.

Suppose first of all that the target particle is at rest in the laboratory(lab) system, so that $\textbf{p}_B=0$ 
and $E_B=m_B$. Then 

$$E^{*2}=-p^2=m_A^2+m_B^2+2m_BE_A$$

Secondly, suppose that the incident and target particles travel in opposite directions. 
Then, with $p_A$ and $p_B$ denoting the absolute values of the 3-momenta, the above 
equation gives 

$$E^{*2}=2(E_AE_B+p_Ap_B)+(m_A^2+m_B^2)\simeq 4E_AE_B$$

if $m_A,m_B\ll E_A,E_B$. This result is for a head-on collision. For two beams crossing at an 
angle $\theta$, the result would be $E^{*2}=2E_AE_B(1+\cos\theta)$.

Note that the cms energy available for new particle creation in a collider with equal energies 
$E$ in the two beams rises linearly with $E$, i.e. $E^*\simeq 2E$, while for a fixed-target 
machine the cms energy rises as the square root of the incident energy, $E^*\simeq\sqrt{2m_BE_A}$. 
Therefore, the highest possible energies for creating new particles are to be found at 
colliding-beam accelerators.

\subsection{The Standard Model of particle physics}

Practically all experimental data from high energy experiments can be accounted for 
by the so-called Standard Model of particles and their interactions.

\subsubsection{The fundamental fermions}

All matter is built from a small number of fundamental spin 1/2 particles, or fermions: 
six quarks and six leptons.

\begin{table}[h]
  \centering
  \caption{The fundamental fermions}
  \begin{tabular}{ccccc}
    \toprule
    Particle & \multicolumn{3}{c}{Flavour} & $Q/\vert e\vert$ \\
    \midrule
    leptons & $e$ & $\mu$ & $\tau$ & -1 \\
            & $\nu_e$ & $\nu_\mu$ & $\nu_\tau$ & 0 \\
    quarks & $u$ & $c$ & $t$ & +2/3 \\
           & $d$ & $s$ & $b$ & -1/3 \\
    \bottomrule
  \end{tabular}
\end{table}

In the table, the quark masses increase from left to right, just as they do for the 
leptons. And, just as for the leptons, the quarks are grouped into pairs differing 
by on unit of electric charge.

The quark type or 'flavour' is denoted by a symbol:
\begin{itemize}
  \item u$\rightarrow$up 
  \item d$\rightarrow$down 
  \item s$\rightarrow$strange 
  \item c$\rightarrow$charmed 
  \item b$\rightarrow$bottom 
  \item t$\rightarrow$top
\end{itemize}

The 's for strange' quark terminology came about because theses quarks turned out to be 
constituents of the so-called 'strange particles' discovered in cosmic rays. Their behaviour 
was strange in the sense that they were produced prolifically in strong interactions, 
and therefore would be expected to decay on a strong interaction timescale($10^{-23}$s); 
instead they decayed extremely slowly, by weak interactions.

The solution to this puzzle was that these particles carried a new quantum number, S for 
strangeness, conserved in strong interactions - so that they were always produced in pairs 
with $S=+1$ and $S=-1$ but they decayed singly and weakly, with a change in strangeness, 
$\Delta S=\pm 1$, into non-strange particles.

A proton consists of $uud$, and a neutron consists of $ddu$. The common material of the present 
universe is the stable particles, i.e. the electrons $e$ and the $u$ and $d$ quarks. The heavier 
quarks $s,c,b,t$ also combine to form particles akin to, but much heavier than, the proton 
and neutron, but these are unstable and decay rapidly(in typically $10^{-13}$s) to $u,d$ 
combinations, just as heavy leptons decay to electrons.

If we allow a new degree of freedom colour(three colours), we'll find that the total charge of all the 
fermions is zero.

\subsubsection{The interactions}

The Standard Model also comprises interactions of particles. There are four types of fundamental interaction or field, as follows.

\begin{enumerate}
  \item \textbf{Strong interactions} are responsible for binding the quarks in the neutron and proton, and the neutrons and protons within nuclei. The interquark force is mediated by a massless particle, the gluon.
  \item \textbf{Electromagnetic interactions} are responsible for virtually all the phenomena in extra-nuclear physics, in particular for the bound states of electrons with nuclei. These interactions are mediated by photon exchange.
  \item \textbf{Weak interactions} are typified by the slow process of nuclear $\beta$-decay. The mediators of the weak interactions are the $W^\pm$ and $Z^0$ bosons, with masses of order 100 times the proton mass.
  \item \textbf{Gravitational interactions} act between all types of particle. On the scale of experiments in particle physics, gravity is by far the weakest of the universe. It is supposedly mediated by exchange of a spin boson, the graviton.
\end{enumerate}

\begin{table}[h]
  \centering
  \caption{The boson mediators}
  \begin{tabular}{lcc}
    \toprule
    Interaction & Mediator & Spin/parity \\
    \midrule
    strong & gluon, $G$ & $1^-$ \\
    electromagnetic & photon, $\gamma$ & $1^-$ \\
    weak & $W^\pm$, $Z^0$ & $1^-$, $1^+$ \\
    gravity & gravition, $g$ & $2^+$ \\
    \bottomrule
  \end{tabular}
\end{table}

Weak and electromagnetic interactions can indeed be unified, and would have the same strength 
at very high energies; only at lower energies is the symmetry broken so that their apparent strengths 
are very different.

To indicate the relative magnitudes of the four types of interaction, the comparative 
strengths of the force between two protons when just in contact are very roughly as follows.

\begin{itemize}
  \item strong: $1$
  \item electromagnetic: $10^{-2}$
  \item weak: $10^{-7}$
  \item gravity: $10^{-39}$
\end{itemize}

As for timescales, the Uncertainty Principle relates the lifetime and the uncertainty 
in energy of a state. An unstable particle does not have a unique mass, but a distribution 
with 'width' $\Gamma=\hbar/\tau$. So, when $\tau$ is very short, its value can be inferred 
from the measured width $\Gamma$.

\subsubsection{Limitations of the Standard Model}

\begin{enumerate}
  \item Gravitational interactions are not included.
  \item Neutrinos are assumed to be massless, but there is growing evidence that neutrinos do have finite masses.
  \item The model is somewhat inelegant, as it contains some 17 arbitrary parameters.
  \item The origin of the parameters and the underlying reasons for the 'xerox copies' - six quark and six lepton flavours - is not at all understood.
\end{enumerate}

We require new and presently unknown physics beyond it. But equally, it seems fairly certain 
that the model will form an integral and important part of a more complete theory of particles in the far future.

\subsection{Particle classification: fermions and bosons}

Fundamental particles are of two types.

\begin{itemize}
  \item \textbf{Fermions}: particles with half-integral spin and obey Fermi-Dirac statistics.
  \item \textbf{Bosons}: particles with integral spin and obey Bose-Einstein statistics.
\end{itemize}

The statistics obeyed by a particle determines how the wavefunction $\psi$ describing 
an ensemble of identical particles behaves under interchange of any pair of particles. 
Clearly $\vert\psi\vert^2$ cannot be altered by the interchange, since particles are 
indistinguishable. Thus, under interchange $\psi\rightarrow\pm\psi$. There is a fundamental 
theorem, which is a sacrosanct principle of quantum field theory.

\begin{itemize}
  \item under exchange of identical bosons $\psi\rightarrow +\psi$; $\psi$ is symmetric
  \item under exchange of identical fermions $\psi\rightarrow -\psi$; $\psi$ is antisymmetric
\end{itemize}

Pauli principle: two or more indentical fermions cannot exist in the same quantum state.

One exciting possible extension beyond the Standard Model is the concept of supersymmetry, which 
predicts that, at a high energy scale there should be fermion-boson symmetry. Each fermion 
will have a boson partner and vice versa.

\subsection{Particles and antiparticles}

The total energy $E$ can in principle assume negative as well as positive values,

$$E=\pm\sqrt{p^2c^2+m^2c^4}$$

Classically, negative energies for free particles appear to be meaningless. In quantum 
mechanics, however, we represent the amplitude of an infinite stream of particles travelling 
along the positive $x$-axis with 3-momentum $p$ by the plane wavefunction 

$$\psi=Ae^{-i(Et-px)/\hbar}$$

where the angular frequency is $\omega=E/\hbar$, the wavenumber is $k=p/\hbar$ and $A$ 
is a normalisation constant.

As $t$ increases, the phase advances in the direction of increasing $x$. Formally, however, 
it can also represent particles of energy $-E$ and momentum $-p$ travelling in the negative $x$-direction 
and backwards in time(i.e. replacing $Et$ by $(-E)(-t)$ and $px$ by $(-p)(-x)$).

Such a stream of negative electrons flowing backwards in time is equivalent to positive charges 
flowing forward, and thus having $E>0$. Hence, the negative energy particle states are connected 
with the existence of positive energy antiparticles of exactly equal but opposite electrical 
charge and magnetic moment, and otherwise identical.

Dirac's original picture of antimatter was that the vacuum actually consisted of an 
infinitely deep sea of completely filled negative energy levels. A positive energy electron 
was prevented from falling into a negative energy state, with release of energy, by the Pauli principle. 
If one supplies energy $E>2mc^2$, a negative energy electron could be lifted into a positive 
energy state. However, such a picture is not valid for the pair creation of bosons.

In the relativistic case, $\psi$ should be treated as an operator that creates or destroys 
particles. Negative energies are simply associated with destruction operators acting on positive 
energy particles to reduce the energy within the system. The absorption or destruction of a 
negative energy particle is again interpreted as the creation of a positive energy antiparticle, 
with opposite charge, and vice versa.

For fermions only there is a conservation law: the difference in the number of fermions 
and antifermions is a constant. Thus fermions and antifermions can only be created or 
destroyed in pairs.

\subsection{Free particle wave equations}

\subsection{Helicity states: helicity conservation}

\subsection{Lepton flavours}

\subsection{Quark flavours}

\subsection{The cosmic connection}

\subsubsection{Early work in cosmic rays}

\subsubsection{Particle physics in cosmology}

\section{Interactions and fields}

\section{Invariance principles and conservation laws}

\section{Quarks in hadrons}

\section{Lepton and quark scattering}

\section{Quark interactions and QCD}

\section{Weak interactions}

\section{Electroweak interactions and the Standard Model}

\section{Physics beyond the Standard Model}

\section{Particle physics and cosmology}

\section{Experimental methods}

\end{document}
